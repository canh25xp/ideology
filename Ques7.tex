\section{Quan điểm của Hồ Chí Minh về vai trò của văn hóa}

\subsection{Phân tích quan điểm của Hồ Chí Minh về vai trò của văn hóa. Liên hệ vai trò của văn hóa với sự phát triển bền vững của Việt Nam hiện nay ?}

\subsubsection{Quan điểm của Hồ Chí Minh về vai trò của văn hóa}

\paragraph{Văn hóa là mục tiêu, động lực của sự nghiệp cách mạng.}
Mục tiêu của Hồ Chí Minh về văn hóa cũng giống như mục tiêu của chủ tịch Hồ Chí Minh, cũng giống như mục tiêu của toàn thể dân tộc Việt Nam: Đó là độc lập dân tộc và CNXH. Tuy nhiên trên lĩnh vực văn hóa, cụ thể hóa ra thành:
\begin{itemize}
    \item Quyền được sống, quyền sung sướng, quyền tự do và quyền mưu cầu hạnh phúc.
    \item Đó là khát vọng của nhân dân về các giá trị: Chân, thiện, mỹ
    \item Đó là xã hội dân chủ, công bằng, văn minh, ai cũng có cơm ăn, áo mặc,. . .
    \item Đời sống vật chất và tinh thần của nhân dân được nâng cao, con người có điều kiện phát triển toàn diện.
\end{itemize} 
Động lực: Văn hóa là động lực cho sự thúc đẩy của kinh tế, chính trị và xã hội. Ở góc độ:
\begin{itemize}
    \item Văn hóa chính trị: Soi đường, mở đường cho quốc dân đi, lãnh đạo nhân dân thực hiện độc lập, tự chủ.
    \item Văn hóa văn nghệ: góp phần nâng cao lòng yêu nước, lý tưởng, tình cảm cách mạng.
    \item Văn hóa giáo dục: Diệt giặc dốt, xóa mù chữ, giúp con người hiểu được các quy luật của xã hội.
    \item Văn hóa đạo đức, lối sống: Nâng cao phẩm chất, phong cách lành mạnh hướng đến chân, thiện, mỹ.
\end{itemize}

\paragraph{Văn hóa là một mặt trận.}
Mặt trận văn hóa là cuộc đấu tranh cách mạng trên lĩnh vực văn hóa.

Đấu tranh trên các lĩnh vực: tư tưởng, đạo đức, lối sống,. . . của các hoạt động văn nghệ, báo chí, công tác lý luận, đặc biệt là định hướng giá trị Chân, Thiện, Mỹ.

Anh em nghệ sĩ là chiến sĩ trên mặt trận - có nhiệm vụ phụng sự tổ quốc và nhân dân.

\paragraph{Văn hóa phục vụ quần chúng nhân dân}
Mọi hoạt động của văn hóa phải trở về cuộc sống thực tại của quần chúng, phản ánh tư tưởng và khát vọng của quần chúng,

Văn hóa phải miêu tả cho hay, cho thật, cho hùng hồn. Phải trả lời được câu hỏi. Viết cho ai? Viết vì mục đích gì? Viết như thế nào?

Viết phải thiết thực, tránh cái lối rau muống. Nói ít, nói cho chắc chắn, thấm thía. . .

\subsubsection{Liên hệ: Vai trò của văn hóa với sự phát triển bền vững của Việt Nam hiện nay:}
\begin{itemize}
    \item Văn hóa giữ vị trí đặc biệt và vai trò quan trọng trong sự điều tiết, vận động mọi mặt của xã hội; là động lực trực tiếp thúc đẩy sự phát triển bền vững kinh tế - xã hội; kích thích sự sáng tạo và đánh thức những năng lực tiềm ẩn của con người. Nhân tố văn hóa không nằm ngoài kinh tế - xã hội hay chính trị, đồng thời là một bộ phận thiết yếu trong đường lối quân sự của chiến lược bảo vệ Tổ quốc Việt Nam. Tóm lại, văn hóa có mặt và giữ vị trí trọng yếu trong mọi lĩnh vực của đời sống xã hội.
    \item Trong điều kiện hiện nay, văn hóa thấm vào mọi mặt của đời sống xã hội, nhất là những người đang giữ vai trò trong sáng tạo khoa học; vào lĩnh vực chính trị với tư cách là văn hóa chính trị; vào kinh tế với tư cách là văn hóa kinh doanh, văn hóa doanh nghiệp và quản trị doanh nghiệp; vào tổ chức, quản trị và điều hành đất nước...
    \item Văn hóa là động lực thúc đẩy kinh tế phát triển. Tác nhân văn hóa trong kinh tế làm cho kinh tế trở thành văn hóa kinh tế. Kinh tế được bảo đảm bởi văn hóa sẽ là kinh tế phát triển, cả trình độ và chất lượng, theo tính nhân văn, vì con người, phục vụ lợi ích của con người, của cộng đồng.
    \item Văn hóa thúc đẩy vai trò của chính trị. Chính trị được bảo đảm bởi văn hóa, văn hóa chứ không đơn thuần là học vấn, sẽ là một nền chính trị nhân văn, vì con người.
    \item Văn hóa giáo dục góp phần nâng cao trình độ văn hóa của nhân dân, giúp cho đất nước phát triển, sánh vai với các cường quốc năm châu.
\end{itemize}

\cleardoublepage