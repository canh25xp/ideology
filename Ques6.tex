\section{Nguyên tắc xây dựng đạo đức cách mạng}

\subsection{Phân tích nguyên tắc xây dựng đạo đức cách mạng và liên hệ với việc rèn luyện đạo đức của sinh viên hiện nay.}

\subsubsection{Phân tích quan điểm của Hồ Chí Minh về các nguyên tắc xây dựng đạo đức cách mạng}

Chủ tịch Hồ Chí Minh luôn đề cao vai trò đạo đức của con người “Thiếu một đức thì không thành người”, “có tài mà không có đức thì là người vô dụng”. Song song với việc học tập rèn luyện, Người luôn nhắc nhở ta phải chú ý tu dưỡng đạo đức bản thân. Theo Hồ Chí Minh muốn thực hiện thành công sự nghiệp cách mạng xã hội chủ nghĩa- cuộc cách mạng sâu sắc nhất triệt để nhất toàn diện nhất chúng ta phải đem hết tinh thần và lực lượng ra phấn đấu, phải tu dưỡng rèn luyện đạo đức cách mạng.

\paragraph{Nói phải đi đôi với làm, nêu gương đạo đức cá nhân}
Hồ Chí Minh coi nguyên tắc nói đi đôi với làm là nguyên tắc quan trọng bậc nhất trong xây dựng nền đạo đức mới. Nguyên tắc cơ bản này là sự thống nhất giữa lý luận thực tiễn, nó đã trở thành phương pháp luận trong cuộc sống là là nền tảng triết lý sống của Người.

Hồ Chí Minh đề cao việc nêu gương đạo đức. Sự gương mẫu của cán bộ , đảng viên trong lời nói và việc làm không chỉ là cách thức để giáo dục đạo đức cho quần chúng mà còn là một phương pháp để tự giáo dục bản thân.

Như vậy, một nền đạo đức mới chỉ được xây dựng trên một nền tảng rộng lớn, vững chắc, khi những chuẩn mực đạo đức trở thành hành vi đạo đức hằng ngày của mỗi người và của toàn xã hội, mà những tấm gương đạo đức của những người tiêu biểu có ý nghĩa thúc đẩy cho quá trình đó.

\paragraph{Xây đi đôi với chống}
Hồ Chí Minh cho rằng, đây là đòi hỏi của nền đạo đức mới thể hiện tính nhân đạo chiến đấu vì mục tiêu của sự nghiệp cách mạng: Xây tức là xây dựng, bồi dưỡng các giá trị, các chuẩn mực đạo đức mới; Chống là chống các biểu hiện, hành vi vô đạo đức, suy thoái đạo đức.

Theo Người, xây dựng nền đạo đức mới phải được tiến hành bằng việc giáo dục những phẩm chất, những chuẩn mực xã hội, nhất là trong những tập thể - nơi mà phần lớn thời gian mỗi con người gắn bó bằng hoạt động thực tiễn của mình.

Việc giáo dục đạo đức phải được tiến hành phù hợp với từng giai đoạn cách mạng cụ thể, phù hợp với từng lứa tuổi, ngành nghề, giai cấp, tầng lớp trong từng môi trường khác nhau. Khơi dậy ý thức đạo đức lành mạnh ở mỗi người.

Để thực hiện xây và chống có hiệu quả, theo Hồ Chí Minh phải tuyên truyền, vận động hình thành phong trào quần chúng rộng rãi đấu tranh cho sự lành mạnh, trong sạch về đạo đức, thôi thúc trách nhiệm đạo đức cá nhân để mọi người phấn đấu tự bồi dưỡng và nâng cao phẩm chất đạo đức cách mạng; phải chú trọng kết hợp giáo dục đạo đức với tăng cường tính nghiêm minh của pháp luật.

\paragraph{Tu dưỡng đạo đức suốt đời}
Theo Người, đó là một quá trình gian khổ, trường kỳ. Một nền đạo đức mới chỉ có thể được xây dựng dựa trên cơ sở tự giác tu dưỡng đạo đức của mỗi người thông qua các hoạt động thực tiễn: Nhìn thẳng vào mình, không tự lừa dối, huyễn hoặc; thấy rõ điểm chưa tốt của mình để khắc phục; kiên trì, tu dưỡng suốt đời.

Người nhấn mạnh mỗi người cần thường xuyên được giáo dục và tự giáo dục về mặt đạo đức. Mỗi người cần luôn bền bỉ, cố gắng. Có rèn luyện như vậy, con người mới có được những phẩm chất tốt đẹp và những phẩm chất ấy ngày được bồi đắp, nâng cao.

\subsubsection{Liên hệ với việc rèn luyện đạo đức của sinh viên hiện nay}

Đạo đức Hồ Chí Minh là đạo đức cách mạng, nêu cao chủ nghĩa tập thể, tiêu diệt chủ nghĩa cá nhân. Để thích ứng với nền kinh tế thị trường, định hướng xã hội chủ nghĩa và hội nhập quốc tế, một nền đạo đức mới đã và đang hình thành. Đó là nền đạo đức vừa phát huy những giá trị truyền thống của dân tộc như yêu nước, thương người, sống nghĩa tình trọn vẹn, cần, kiệm, liêm, chính, chí công vô tư với những yêu cầu mới, những nội dung mới. Do đòi hỏi của dân tộc và thời đại, nhờ đó phần lớn sinh viên vẫn giữ được lối sống tình nghĩa, trong sạch, lành mạnh, khiêm tốn, luôn cần cù và sáng tạo trong học tập, sống có bản lĩnh, có chí lập thân, lập nghiệp, năng động nhạy bén, dám đối mặt với những khó khăn thách thức.

Bên cạnh đó do ảnh hưởng của nền kinh tế thị trường, hội nhập quốc tế, do sự bùng phát của lối sống thực dụng, chạy theo danh lợi, bất chấp đạo lý đã dẫn đến những tiêu cực trong xã hội như đã có một bộ phận thanh niên, trong đó có không ít sinh viên phai nhạt niềm tin, lý tưởng, mất phương hướng phấn đấu, không có chí lập thân, lập nghiệp; chạy theo lối sống thực dụng, sống thử, sống dựa dẫm, thiếu trách nhiệm, thờ ơ với gia đình và xã hội, sa vào nghiện ngập, hút sách, thiếu trung thực, gian lận trong thi cử, chạy điểm, chạy thầy, chạy trường, mua bằng cấp. Đây là những thách thức rất lớn, làm cản trở thanh niên phát triển bản thân, phát triển đất nước.

Sinh viên là thế hệ tương lai của đất nước, chúng ta không chỉ luôn cố gắng nỗ lực trong học tập , sáng tạo tìm tòi cái mới mà song song với đó cần luôn luôn tu dưỡng đạo đức bản thân. Muốn vậy, ta cần phải: thấm nhuần tư tưởng đạo đức HCM thông qua các bài học trên lớp, các phương tiện truyền thông. Dành thời gian thỏa đáng để tìm hiểu đạo đức, tư tưởng HCM về tinh thần, trách nhiệm, nói đi đôi với làm; thực hành tư tưởng đạo đức HCM, tự soi mình, sửa mình, rèn luyện các phẩm chất đạo đức; tích cực tham gia các phong trào của Đoàn TN, Hội SV, phấn đấu rèn luyện đạt danh hiệu SV5T, phấn đấu trở thành tấm gương sáng về tinh thần trách nhiệm, tính trung thực, nói đi đôi với làm, xây đi đôi với chống; tu dưỡng đạo đức suốt đời, không chỉ giới hạn trong một giai đoạn của cuộc đời, rèn luyện đạo đức hàng ngày và trong mọi hoạt động thực tiễn của bản thân bởi lẽ phải đấu tranh rèn luyện bền bỉ thì mới thành được.

\cleardoublepage