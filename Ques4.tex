\section{Tư tưởng Hồ Chí Minh về nhà nước: Nhà nước pháp quyền, Nhà nước dân chủ}

\subsection{Phân tích tư tưởng Hồ Chí Minh về xây dựng nhà nước pháp quyền}

\paragraph{Thứ nhất, nhà nước CHXHCN Việt Nam là nhà nước hiến pháp, hợp pháp.}
Nhà nước ta được nhân dân tổ chức thông qua tổng tuyển cử, được xây dựng và hoạt động theo các nguyên tắc của hiến pháp.

Hồ Chí Minh nêu ra và thực hiện điều này ngay sau khi Cách mạng Tháng Tám thành công, chính quyền thuộc về nhân dân. Sau khi giành chính quyền trong cả nước, Hồ Chí Minh đã thay mặt chính phủ lâm thời đọc bản Tuyên ngôn độc lập tuyên bố với quốc dân đồng bào và thế giới về sự khai sinh của nước Việt Nam mới. Nhờ đó, chính phủ lâm thời do cuộc cách mạng của nhân dân ta lập ra có được địa vị hợp pháp.

Tiếp đó, trong phiên họp đầu tiên của chính phủ lâm thời (ngày 3-9-1945), Hồ Chí Minh đã đề nghị Chính phủ tổ chức sớm cuộc tổng tuyển cử với chế độ phổ thông đầu phiếu để lập nên Quốc hội rồi từ đó lập ra Chính phủ và các cơ quan, bộ máy hợp hiến thể hiện quyền lực tối cao của nhân dân. Ngày 2-3-1946, Quốc hội khoá I nước Việt Nam Dân chủ Cộng hoà đã họp phiên đầu tiên, lập ra các tổ chức, bộ máy và các chức vụ chính thức của nhà nước. Đây chính là Chính phủ có đầy đủ tư cách pháp lí để giải quyết một cách có hiệu quả những vấn đề đối nội và đối ngoại của nước ta.

\paragraph{Thứ hai, nhà nước ta là nhà nước thượng tôn pháp luật.}
Hồ Chí Minh luôn chú trọng vấn đề xây dựng nền tảng pháp lý cho Nhà nước Việt Nam mới. Người đã sớm thấy rõ tầm quan trọng của Hiến pháp và pháp luật trong đời sống chính trị - xã hội. Điều này thể hiện trong bản Yêu sách của nhân dân An Nam: đòi quyền bình đẳng về chế độ pháp lý cho người An Nam. Sau này, khi trở thành người đứng đầu Nhà nước Việt Nam mới, Người càng quan tâm sâu sắc hơn việc bảo đảm cho Nhà nước được tổ chức và vận hành phù hợp với pháp luật, đồng thời, căn cứ vào pháp luật để điều hành xã hội.

Trong tư tưởng Hồ Chí Minh, Nhà nước quản lí xã hội bằng bộ máy nhà nước và nhiều biện pháp khác nhau, nhưng quan trọng nhất là quản lý bằng Hiến pháp và pháp luật nói chung. Muốn vậy, trước hết, cần làm tốt công tác lập pháp.

Cùng với công tác lập pháp, chủ tịch Hồ Chí Minh cũng rất chú trọng đưa pháp luật vào trong cuộc sống, đảm bảo cho pháp luật được thi hành và có cơ chế giám sát việc thi hành pháp luật. Người chỉ rõ sự cần thiết phải nâng cao trình độ hiểu biết và năng lực sử dụng pháp luật cho người dân, giáo dục ý thức tôn trọng và tuân thủ pháp luật trong nhân dân. Người cho rằng công tác giáo dục pháp luật cho mọi người, đặc biệt là thế hệ trẻ trở nên cực kì quan trọng trong việc xây dựng một Nhà nước pháp quyền, đảm bảo mọi quyền và nghĩa vụ của công dân được thực thi trong cuộc sống.

Hồ Chí Minh luôn nêu cao tính nghiêm minh của pháp luật. Điều đó đòi hỏi pháp luật phải đúng và phải đủ; tăng cường tuyên truyền, giáo dục pháp luật cho mọi người dân; người thực thi pháp luật phải thật sự công tâm, nghiêm minh.

Người còn luôn luôn khuyến khích nhân dân phê bình, giám sát công việc của nhà nước, giám sát quá trình nhà nước thực thi pháp luật, đồng thời không ngừng nhắc nhở cán bộ các cấp, các ngành phải gương mẫu trong việc thực thi pháp luật, trước hết là các cán bộ thuộc ngành hành pháp và tư pháp.

\paragraph{Thứ ba, nhà nước ta là nhà nước pháp quyền nhân nghĩa.}
Pháp quyền nhân nghĩa tức là trước hết Nhà nước phải tôn trọng, đảm bảo thực hiện đầy đủ các quyền con người, chăm lo đến lợi ích hợp pháp của mọi người. Người đề cập đến các quyền tự nhiên của con người, trong đó có quyền cao nhất là quyền sống, đồng thời cũng đề cập đến các quyền chính trị - dân sự, quyền kinh tế, văn hoá, xã hội của con người. Hiến pháp của đất nước đã ghi nhận một cách toàn diện quyền con người ở Việt Nam. Đó là nền tảng pháp lý để bảo vệ và thực thi các quyền con người đó một cách triệt để.

Trong pháp quyền nhân nghĩa, pháp luật có tính nhân văn. khuyến thiện. Tính nhân văn của hệ thống pháp luật thể hiện ở việc ghi nhận đầy đủ và bảo vệ quyền con người, ở tính nghiêm minh nhưng khách quan và công bằng. Đặc biệt, hệ thống pháp luật đó có tinh khuyến thiện, bảo vệ cái đúng, cái tốt, lấy mục đích giáo dục, cảm hoá, thức tỉnh con người làm căn bản. Nói cách khác, pháp luật trong Nhà nước pháp quyền nhân nghĩa phải là pháp luật vì con người.

\paragraph{Liên hệ với việc xây dựng nhà nước XHCN Việt Nam hiện nay:}
Đại hội Đảng lần thứ XIII đã xác định: "Tiếp tục xây dựng và hoàn thiện Nhà nước pháp quyền XHCN Việt Nam của nhân dân, do nhân dân, vì nhân dân Đảng lãnh đạo là nhiệm vụ trọng tâm của đổi mới hệ thống chính trị."

Cần đẩy mạnh việc hoàn thiện pháp luật gắn với tổ chức thi hành pháp luật nhằm nâng cao hiệu lực, hiệu quả của Nhà nước; bảo đảm pháp luật vừa là công cụ để Nhà nước quản lí xã hội, vừa là công cụ để nhân dân làm chủ, kiểm tra, giám sát quyền lực nhà nước. Quản lý nhà nước theo pháp luật, đồng thời coi trọng xây dựng nền tảng đạo đức xã hội.

Cần tiếp tục hoàn thiện hệ thống pháp luật, tôn trọng, bảo đảm, bảo vệ quyền con người, quyền và nghĩa vụ của công dân. Phải xác định rõ cơ chế phân công, phối hợp thực thi quyền lực nhà nước, nhất là cơ chế kiểm soát quyền lực giữa các cơ quan nhà nước trong việc thực hiện các quyền lập pháp, hành pháp, tư pháp trên cơ sở kiểm soát quyền lực nhà nước là thống nhất; xác định rõ hơn quyền hạn và trách nhiệm của mỗi quyền.

Chú trọng công tác xây dựng đội ngũ cán bộ, công chức. Đẩy mạnh dân chủ hoá công tác cán bộ, quy định rõ trách nhiệm, thẩm quyền của mỗi tổ chức, mỗi cấp trong xây dựng đội ngũ cán bộ, công chức có bản lĩnh chính trị vững vàng, phẩm chất đạo đức trong sáng, cố trình độ, năng lực chuyên môn phù hợp để thực thi đầy đủ trách nhiệm công vụ, đáp ứng yêu cầu của giai đoạn mới.

\cleardoublepage

\subsection{Phân tích tư tưởng Hồ Chí Minh về xây dựng nhà nước dân chủ}

\paragraph{Bản chất giai cấp của nhà nước}
Trong tư tưởng Hồ Chí Minh, Nhà nước Việt Nam là nhà nước dân chủ, nhưng tuyệt nhiên nó không phải là “Nhà nước toàn dân”, hiểu theo nghĩa là nhà nước phi giai cấp. Nhà nước ở đâu và bao giờ cũng mang bản chất của một giai cấp nhất định. Nhà nước Việt Nam mới – Nhà nước Việt Nam Dân chủ Cộng hòa, theo quan điểm của Hồ Chí Minh, là một nhà nước mang bản chất giai cấp công nhân. Bản chất giai cấp công nhân của Nhà nước Việt Nam thể hiện trên mấy phương diện:

Một là, Đảng Cộng sản Việt Nam giữ vị trí và vai trò cầm quyền. Lời nói đầu của bản Hiến pháp năm 1959 khẳng định: “Nhà nước của ta là Nhà nước dân chủ nhân dân, dựa trên nền tảng liên minh công nông, do giai cấp công nhân lãnh đạo”. Ngay trong quan điểm về nhà nước dân chủ, nhà nước do nhân dân là người chủ nắm chính quyền, Hồ Chí Minh đã nhấn mạnh nòng cốt của nhân dân là liên minh công - nông - trí, do giai cấp công nhân mà đội tiên phong của nó là Đảng Cộng sản Việt Nam lãnh đạo. Đảng cầm quyền bằng phương thức thích hợp sau đây: (1) Bằng đường lối, quan điểm, chủ trương để Nhà nước thể chế hóa thành pháp luật, chính sách, kế hoạch; (2) Bằng hoạt động của các tổ chức đảng và đảng viên của mình trong bộ máy, cơ quan nhà nước; (3) Bằng công tác kiểm tra.

Hai là, bản chất giai cấp của Nhà nước Việt Nam thể hiện ở tính định hướng xã hội chủ nghĩa trong sự phát triển đất nước. Đưa đất nước đi lên chủ nghĩa xã hội và chủ nghĩa cộng sản là mục tiêu cách mạng nhất quán của Hồ Chí Minh. Việc giành lấy chính quyền, lập nên Nhà nước Việt Nam mới, chính là để giai cấp công nhân và nhân dân lao động có được một tổ chức mạnh mẽ nhằm thực hiện mục tiêu nói trên.

Ba là, bản chất giai cấp công nhân của Nhà nước thể hiện ở nguyên tắc tổ chức và hoạt động của nó là nguyên tắc tập trung dân chủ. Hồ Chí Minh rất chú ý đến cả hai mặt dân chủ và tập trung trong tổ chức và hoạt động của tất cả bộ máy, cơ quan nhà nước. Người nhấn mạnh đến sự cần thiết phải phát huy cao độ dân chủ, đồng thời cũng nhấn mạnh phải phát huy cao độ tập trung, Nhà nước phải tập trung thống nhất quyền lực để tất cả mọi quyền lực thuộc về nhân dân.

Trong Nhà nước Việt Nam, bản chất giai cấp công nhân thống nhất với tính nhân dân và tính dân tộc. Hồ Chí Minh là người giải quyết rất thành công mối quan hệ giữa vấn đề dân tộc với vấn đề giai cấp trong cách mạng Việt Nam. Trong tư tưởng của Người về Nhà nước mới ở Việt Nam, bản chất giai cấp công nhân của Nhà nước thống nhất với tính nhân dân và tính dân tộc, thể hiện cụ thể như sau:

Một là, Nhà nước Việt Nam ra đời là kết quả của cuộc đấu tranh lâu dài, gian khổ của rất nhiều thế hệ người Việt Nam, của toàn thể dân tộc. Từ giữa thế kỷ XIX, khi đất nước bị ngoại xâm, các tầng lớp nhân dân Việt Nam, hết thế hệ này đến đến thế hệ khác đã không quản hy sinh, xương máu chiến đấu cho độc lập, tự do của Tổ quốc. Từ khi Đảng Cộng sản Việt Nam ra đời, trở thành lực lượng lãnh đạo sự nghiệp cách mạng của dân tộc, với chiến lược đại đoàn kết đúng đắn, sức mạnh của toàn dân tộc đã được tập hợp và phát huy cao độ, chiến thắng ngoại xâm, giành lại độc lập, tự do, lập nên Nhà nước Việt Nam Dân chủ Cộng hòa - Nhà nước dân chủ nhân dân đầu tiên ở Đông Nam châu Á. Nhà nước Việt Nam mới, do vậy, không phải của riêng giai cấp, tầng lớp nào, mà là thuộc về nhân dân.

Hai là, Nhà nước Việt Nam ngay từ khi ra đời đã xác định rõ và luôn kiên trì, nhất quán mục tiêu vì quyền lợi của nhân dân, lấy quyền lợi của dân tộc làm nền tảng. Bản chất của vấn đề này là ở chỗ, Hồ Chí Minh khẳng định quyền lợi cơ bản của giai cấp công nhân thống nhất với lợi ích của nhân dân lao động và của toàn dân tộc. Nhà nước Việt Nam mới là người đại diện, bảo vệ, đấu tranh không chỉ cho lợi ích của giai cấp công nhân, mà còn của nhân dân lao động và của toàn dân tộc.

Ba là, trong thực tế, Nhà nước mới ở Việt Nam đã đảm đương nhiệm vụ mà toàn thể dân tộc giao phó là tổ chức nhân dân tiến hành các cuộc kháng chiến để bảo vệ nền độc lập, tự do của Tổ quốc, xây dựng một nước Việt Nam hòa bình, thống nhất, độc lập, dân chủ và giàu mạnh, góp phần tích cực vào sự phát triển tiến bộ của thế giới. Con đường quá độ lên chủ nghĩa xã hội và đi đến chủ nghĩa cộng sản là con đường mà Hồ Chí Minh và Đảng ta đã xác định, cũng là sự nghiệp của chính Nhà nước.

\paragraph{Nhà nước của nhân dân}
Theo quan điểm của Hồ Chí Minh, nhà nước của nhân dân là nhà nước mà tất cả mọi quyền lực trong nhà nước và trong xã hội đều thuộc về nhân dân. Người khẳng định: “Trong Nhà nước Việt Nam Dân chủ Cộng hòa của chúng ta, tất cả mọi quyền lực đều là của nhân dân”. Nhà nước của dân tức là “dân là chủ”. Nguyên lý “dân là chủ” khẳng định địa vị chủ thể tối cao của mọi quyền lực là nhân dân.

Trong Nhà nước dân chủ, nhân dân thực thi quyền lực thông qua hai hình thức dân chủ trực tiếp và dân chủ gián tiếp. Dân chủ trực tiếp là hình thức dân chủ trong đó nhân dân trực tiếp quyết định mọi vấn đề liên quan đến vận mệnh của quốc gia, dân tộc và quyền lợi của dân chúng. Hồ Chí Minh luôn coi trọng hình thức dân chủ trực tiếp bởi đây là hình thức dân chủ hoàn bị nhất, đồng thời tạo mọi điều kiện thuận lợi để thực hành dân chủ trực tiếp.

Cùng với dân chủ trực tiếp, dân chủ gián tiếp hay dân chủ đại diện là hình thức dân chủ được sử dụng rộng rãi nhằm thực thi quyền lực của nhân dân. Đó là hình thức dân chủ mà trong đó nhân dân thực thi quyền lực của mình thông qua các đại diện mà họ lựa chọn, bầu ra và những thiết chế quyền lực mà họ lập nên. Theo quan điểm của Hồ Chí Minh, trong hình thức dân chủ gián tiếp:

Quyền lực nhà nước là “thừa ủy quyền” của nhân dân. Tự bản thân nhà nước không có quyền lực. Quyền lực của nhà nước là do nhân dân ủy thác do. Do vậy, các cơ quan quyền lực nhà nước cùng với đội ngũ cán bộ của nó đều là “công bộc” của nhân dân, nghĩa là “gánh vác việc chung cho dân, chứ không phải để đè đầu dân”. Ở đây, Hồ Chí Minh đã xác định rõ vị thế và mối quan hệ giữa nhân dân với cán bộ nhà nước trên cơ sở nhân dân là chủ thể nắm giữ mọi quyền lực. Theo Hồ Chí Minh: “Dân làm chủ thì Chủ tịch, Bộ trưởng, thứ trưởng, uỷ viên này uỷ viên khác là làm gì? Làm đày tớ. Làm đày tớ cho nhân dân, chứ không phải là làm quan cách mạng”; “Nước ta là nước dân chủ, địa vị cao nhất là dân, vì dân là chủ. Trong bộ máy cách mạng, từ người quét nhà, nấu ăn cho đến Chủ tịch một nước đều là phân công làm đầy tớ cho dân”. Hồ Chí Minh kịch liệt phê phán những cán bộ nhà nước thoái hóa, biến chất, từ chỗ là công bộc của dân đã trở thành “quan cách mạng”, đứng trên nhân dân, coi khinh nhân dân, “cậy thế” với dân, “quên rằng dân bầu mình ra là để làm việc cho dân”.

Nhân dân có quyền kiểm soát, phê bình nhà nước, có quyền bãi miễn những đại biểu mà họ đã lựa chọn, bầu ra và có quyền giải tán những thiết chế quyền lực mà họ đã lập nên. Đây là quan điểm rõ ràng, kiên quyết của Hồ Chí Minh nhằm đảm bảo cho mọi quyền lực, trong đó có quyền lực nhà nước, luôn nằm trong tay dân chúng. Một nhà nước thật sự của dân, theo Hồ Chí Minh, luôn “mong đồng bào giúp đỡ, đôn đốc, kiểm soát và phê bình để làm trọn nhiệm vụ của mình là người đầy tớ trung thành tận tuỵ của nhân dân”; trong Nhà nước đó, “nhân dân có quyền bãi miễn đại biểu Quốc hội và đại biểu Hội đồng nhân dân nếu những đại biểu ấy tỏ ra không xứng đáng với sự tín nhiệm của nhân dân”, thậm chí, “nếu Chính phủ làm hại dân thì dân có quyền đuổi Chính phủ”.

Luật pháp dân chủ và là công cụ quyền lực của nhân dân. Theo Hồ Chí Minh, sự khác biệt căn bản của luật pháp trong Nhà nước Việt Nam mới với luật pháp của các chế độ tư sản, phong kiến là ở chỗ nó phản ánh được ý nguyện và bảo vệ quyền lợi của dân chúng. Luật pháp đó là của nhân dân, là công cụ thực thi quyền lực của nhân dân, là phương tiện để kiểm soát quyền lực nhà nước.

\paragraph{Nhà nước do nhân dân}
Trong tư tưởng Hồ Chí Minh, nhà nước do nhân dân trước hết là nhà nước do nhân dân lập nên sau thắng lợi của sự nghiệp cách mạng của toàn dân tộc dưới sự lãnh đạo của Đảng Cộng sản Việt Nam. Nhân dân “cử ra”, “tổ chức nên” nhà nước dựa trên nền tảng pháp lý của một chế độ dân chủ và theo các trình tự dân chủ với các quyền bầu cử, phúc quyết, v.v..

Nhà nước do nhân dân còn có nghĩa “dân làm chủ”. Người khẳng định rõ: “Nước ta là nước dân chủ, nghĩa là nước nhà do nhân dân làm chủ”. Nếu “dân là chủ” xác định vị thế của nhân dân đối với quyền lực nhà nước, thì “dân làm chủ” nhấn mạnh quyền lợi và nghĩa vụ của nhân dân với tư cách là người chủ. Theo quan điểm của Hồ Chí Minh, “nhân dân có quyền lợi làm chủ, thì phải có nghĩa vụ làm tròn bổn phận công dân, giữ đúng đạo đức công dân”. Nhân dân làm chủ thì phải tuân theo pháp luật của Nhà nước, tuân theo kỷ luật lao động, giữ gìn trật tự chung, đóng góp (nộp thuế) đúng kỳ, đúng số để xây dựng lợi ích chung, hăng hái tham gia công việc chung, bảo vệ tài sản công cộng, bảo vệ Tổ quốc, v.v..

Trong nhà nước do nhân dân làm chủ, nhà nước phải tạo mọi điều kiện để nhân dân được thực thi những quyền mà Hiến pháp và pháp luật đã quy định, hưởng dụng đầy đủ quyền lợi và làm tròn nghĩa vụ làm chủ của mình. Người yêu cầu cán bộ, đảng viên phải thật sự tôn trọng quyền làm chủ của nhân dân.

Nhà nước do nhân dân cần coi trọng việc giáo dục nhân dân, đồng thời nhân dân cũng phải tự giác phấn đấu để có đủ năng lực thực hiện quyền dân chủ của mình. Hồ Chí Minh nói: “Chúng ta là những người lao động làm chủ nước nhà. Muốn làm chủ được tốt, phải có năng lực làm chủ”. Không chỉ tuyên bố quyền làm chủ của nhân dân, cũng không chỉ đưa nhân dân tham gia công việc nhà nước, mà còn chuẩn bị và động viên nhân dân chuẩn bị tốt năng lực làm chủ, quan điểm đó thể hiện tư tưởng dân chủ triệt để của Hồ Chí Minh khi nói về nhà nước do nhân dân.

\paragraph{Nhà nước vì nhân dân}
Nhà nước vì dân là nhà nước phục vụ lợi ích và nguyện vọng của nhân dân, không có đặc quyền đặc lợi, thực sự trong sạch, cần kiệm liêm chính. Hồ Chí Minh là một vị Chủ tịch vì dân và Người yêu cầu các cơ quan nhà nước, các cán bộ nhà nước đều phải vì nhân dân phục vụ. Người nói: "Các công việc của Chính phủ làm phải nhằm vào một mục đích duy nhất là mưu tự do hạnh phúc cho mọi người. Cho nên Chính phủ nhân dân bao giờ cũng phải đặt quyền lợi dân lên trên hết thảy. Việc gì có lợi cho dân thì làm. Việc gì có hại cho dân thì phải tránh”. Theo Hồ Chí Minh, thước đo một Nhà nước vì dân là phải được lòng dân. Hồ Chí Minh đặt vấn đề với cán bộ Nhà nước phải “làm sao cho được lòng dân, dân tin, dân mến, dân yêu”, đồng thời chỉ rõ: “muốn được dân yêu, muốn được lòng dân, trước hết phải yêu dân, phải đặt quyền lợi của dân trên hết thảy, phải có một tinh thần chí công vô tư”. Trong Nhà nước vì dân, cán bộ vừa là đầy tớ, nhưng đồng thời phải vừa là người lãnh đạo nhân dân. Hai đòi hỏi này tưởng chừng như mâu thuẫn, nhưng đó là những phẩm chất cần có ở người cán bộ nhà nước vì dân. Là đầy tớ thì phải trung thành, tận tụy, cần kiệm liêm chính, chí công vô tư, lo trước thiên hạ, vui sau thiên hạ. Là người lãnh đạo thì phải có trí tuệ hơn người, minh mẫn, sáng suốt, nhìn xa trông rộng, gần gũi nhân dân, trọng dụng hiền tài. Như vậy, để làm người thay mặt nhân dân phải gồm đủ cả đức và tài, phải vừa hiền lại vừa minh. Phải như thế thì mới có thể “chẳng những làm những việc trực tiếp có lợi cho dân, mà cũng có khi làm những việc mới xem qua như là hại đến dân”, nhưng thực chất là vì lợi ích toàn cục, vì lợi ích lâu dài của nhân dân.

\cleardoublepage