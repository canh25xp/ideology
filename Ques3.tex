\section{Những vấn đề nguyên tắc trong hoạt động của Đảng: Đảng lấy chủ nghĩa Mác-Lênin làm nền tảng tư tưởng.}

\subsection{Tại sao Đảng cộng sản Việt Nam phải lấy chủ nghĩa Mác-Lênin làm nền tảng theo Hồ Chí Minh}

\subsubsection{Đảng cộng sản Việt Nam phải lấy chủ nghĩa Mác-Lênin làm nền tảng theo Hồ Chí Minh}

Vì chủ nghĩa Mác-Lênin là nền tảng tư tưởng và kim chỉ nam cho hành động của Đảng. Nói về tầm quan trọng của lí luận, trong tác phẩm Đường Kách mệnh (1927), Hồ Chí Minh khẳng định: "Đảng muốn vững thì phải có chủ nghĩa làm cốt, trong Đảng ai cũng phải hiểu, ai cũng phải theo chủ nghĩa ấy. Đảng mà không có có chủ nghĩa cũng như người không có trí khôn, tàu không có bàn chỉ nam. Bây giờ học thuyết nhiều, chủ nghĩa nhiều, nhưng chủ nghĩa chân chính nhất, chắc chắn nhất, cách mệnh nhất là chủ nghĩa Lênin."

Hồ Chí Minh luôn luôn nhấn mạnh phải trung thành với chủ nghĩa Mác-Lênin nhưng đồng thời phải biết sáng tạo, vận dụng cho phù hợp với điều kiện, hoàn cảnh, từng lúc từng nơi, không được phép giáo điều.

\paragraph{Đối với cơ quan và tổ chức Đảng, Nhà nước, Hồ Chí Minh yêu cầu:}
\begin{itemize}
    \item Chủ nghĩa Mác - Lênin là kim chỉ nam cho hành động chứ không phải là kinh thánh. Phải nắm vững phép duy vật biện chứng và duy vật lịch sử, phải căn cứ vào thực tiễn để xác định đúng quan điểm, đường lối chính sách, pháp luật; phải vận dụng sáng tạo, không máy móc, giáo điều, rập khuôn.
    \item Phải tìm các giải pháp, biện pháp thực hiện, phải có quyết tâm, "đường lối 1, biện pháp phải 10, quyết tâm phải 20".
    \item Phải tổng kết kinh nghiệm của Đảng minh, các Đảng anh em, tổng kết thực tiễn để rút ra những bài học thành công, chưa thành công.
    \item Phải tổ chức việc học tập, bồi dưỡng lý luận cho cán bộ, đảng viên phủ hợp, đồng thời phải tuyên truyền vận động quần chúng thực hiện đường lối, chính sách.
\end{itemize}

\paragraph{Đối với cán bộ, đảng viên, Hồ Chí Minh yêu cầu:}
\begin{itemize}
    \item Học lý luận, hiểu lý luận là phải vận dụng vào thực tiễn, "học mà không hành là cái hòm đựng sách'', ``để lòe thiên hạ'', để ra vẻ ta đây''. Phải chống các biểu hiện kém lý luận, coi thường lý luận, lý luận suông, lý luận giáo điều.
    \item Phải tin tưởng nhất trí với quan điểm, đường lối của Đảng, Nhà nước và quyết tâm thực hiện; phải bảo vệ quan điểm đường lối chính sách, giữ vững kỷ luật Đảng và kỷ luật cơ quan đoàn thể.
    \item Phải lấy hiệu quả công tác, hoàn thành nhiệm vụ làm thước đo sự hiểu và vận dụng chủ nghĩa Mác – Lênin; công việc bê trễ thì không thể nói là hiểu chủ nghĩa Mác – Lênin được.
    \item Phải coi việc thường xuyên học tập lý luận là nhiệm vụ và tiêu chuẩn đảng viên.
    \item Phải sống với nhau có tình có nghĩa.
\end{itemize}

\subsubsection{Liên hệ vai trò của sinh viên trong việc bảo vệ nền tảng tư tưởng của Đảng}

Bản thân mỗi sinh viên, đặc biệt là sinh viên BKHN phải nhận thức rõ nền tảng tư tưởng của Đảng là chủ nghĩa Mác-Lênin và tư tưởng Hồ Chí Minh. Sinh viên cần xây dựng cơ sở khoa học cho nhận thức và niềm tin vững chắc vào con đường đi lên của chủ nghĩa xã hội. Nhiều sinh viên đã chủ động đấu tranh với những luận điệu xuyên tạc, thông tin xấu độc, góp phần bảo vệ nền tảng tư tưởng của Đảng.

Trong bối cảnh quốc tế diễn biến phức tạp, tình hình trong nước nhiều khó khăn, các thế lực thù địch để tuyên truyền những quan điểm sai trái đã sử dụng rất nhiều thủ đoạn, phương thức khác nhau, ngày càng tinh vi, nguy hiểm hơn, có tác động đến các tầng lớp trong đó thanh niên là đối tượng chủ yếu bị tấn công. Thực tiễn cho thấy, không ít thanh niên đã có những biểu hiện phai nhạt lý tưởng, xa rời mục tiêu. Hiện nay, sự bùng nổ của Internet và mạng xã hội tác động mạnh mẽ đến cuộc sống chúng ta, nó đặt ra nhiều vấn đề là những đạo lý, văn hoá, bản sắc dân tộc đang bị xâm hại. Do đó cần có giải pháp chỉnh đốn đồng thời phát huy cái hay cái đẹp của dân tộc ta trên con đường hội nhập.

Để bảo đảm nền tảng tư tưởng của Đảng, mỗi sinh viên phải tích cực học tập, nâng cao trình độ kiến thức, năng lực, phẩm chất chính trị, đạo đức, có lối sống lành mạnh; đấu tranh chống chủ nghĩa cá nhân, cơ hội, cảnh giác trước những âm mưu, thủ đoạn chống phá của các đối tượng xấu. Trên ``mặt trận'' không gian mạng, mỗi thanh niên phải tận dụng sức trẻ, khả năng sáng tạo để tạo nên những sản phẩm thông tin về bảo vệ nền tảng tư tưởng của Đảng một cách đa dạng, mới mẻ, thu hút đông đảo công chúng. Phải kịp thời phát hiện, cảnh báo cho cộng đồng về các trang cung cấp thông tin xấu, độc, giả mạo; các nguy cơ xâm hại đến nền tảng tư tưởng của Đảng, từ đó phối hợp các cơ quan chức năng xử lý các hành vi vi phạm pháp luật.

\cleardoublepage