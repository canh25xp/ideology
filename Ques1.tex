\section{Cơ sở lý luận hình thành tư tưởng Hồ Chí Minh: Tinh hoa văn hóa phương Đông, tinh hoa văn hóa phương Tây.}

\subsection{Phân tích tinh hoa văn hóa Phương Đông trong việc hình thành Tư tưởng Hồ Chí Minh, từ đó rút ra giá trị của tinh hoa ấy trong giai đoạn hiện nay?}

\subsubsection{Phân tích tinh hoa văn hóa Phương Đông trong việc hình thành Tư tưởng Hồ Chí Minh}
Chủ tịch Hồ Chí Minh đã tiếp thu một cách chọn lọc những tư tưởng văn hóa tiến bộ của Phương Đông, có thể kể tới như Nho giáo, Phật giáo, Lão giáo, hay một số tư tưởng tiến bộ như chủ nghĩa Tam Dân của Tôn Trung Sơn. Cụ thể:

\paragraph{Nho giáo:}
Hồ Chí Minh đã kế thừa và phát triển những mặt tích cực của Nho giáo như: tư tưởng nhân trị và đức trị để quản lý xã hội. Kế thừa và phát triển quan niệm của Nho giáo về việc xây dựng một xã hội lý tưởng trong đó công bằng, bác ái, nhân, nghĩa, trí, dũng, tín, liêm được coi trọng để có thể đi đến một thế giới đại đồng với hòa bình, không có chiến tranh, các dân tộc có quan hệ hữu nghị và hợp tác. Đặc biệt, Hồ Chí Minh chú ý kế thừa, đổi mới, phát tinh thần trong đạo đức của Nho giáo trong việc tu dưỡng rèn luyện đạo đức con người,trong công tác xây dựng Đảng về đạo đức. Tuy nhiên, Hồ Chí Minh cũng phê phán các mặt tiêu cực của Nho giáo như: bảo vệ chế độ phong kiến, phân chia đẳng cấp, trọng nam khinh nữ, coi trọng thi cử; suy nghĩ bảo thủ, một chiều; khinh thường công việc lao động chân tay, lao động xã hội, chỉ đề cao nghề đọc
sách...

Người tiếp thu thông qua việc thường xuyên sử dụng các mệnh đề để đưa ra vấn đề về việc tu dưỡng đạo đức cá nhân (thể hiện xuyên suốt trong các tác phẩm ``Đường Cách mệnh'' (1927), ``Sửa đổi lối làm việc'' (1947), ``Cần, kiệm, liêm, chính'' (1949) cho đến ``Nâng cao đạo đức cách mạng, quét sạch chủ nghĩa cá nhân'' (1969) và bản Di chúc của Người
(1969)).

\paragraph{Phật giáo:}
Hồ Chí Minh chú ý kế thừa, phát triển tư tưởng từ bi, vị tha, yêu thương con người, khuyến khích làm việc thiện, chống lại điều ác; đề cao quyền bình đẳng của con người và chân lý; khuyên con người sống hòa đồng, gắn bó với đất nước của Đạo Phật. Đồng thời phê phán thế giới quan duy tâm, tư tưởng an phận, bi quan yếm thế. Những quan điểm tích cực trong triết lý của Đạo Phật đã được Hồ Chí Minh vận dụng sáng tạo để đoàn kết đồng bào theo đạo Phật, đoàn kết toàn dân vì một nước Việt Nam hòa bình, thống nhất, độc lập, dân chủ và giàu mạnh.

\paragraph{Lão giáo:}
Hồ Chí Minh chú ý kế thừa, phát triển tư tưởng của Lão Tử, khuyên con người nên sống gắn bó với thiên nhiên, hòa đồng với thiên nhiên, hơn nữa phải biết bảo vệ môi trường sống. Người khuyên cán bộ, đảng viên ít lòng ham muốn về vật chất; thực hiện cần kiệm liêm chính, chí công vô tư; hành động theo đạo lý với ý nghĩa là hành động đúng với quy luật tự nhiên, xã hội.

\paragraph{Chủ nghĩa Tam dân:}
Người tiếp tục tìm hiểu chủ nghĩa Tam Dân: ``Dân tộc độc lập, dân quyền tự do và dân sinh hạnh phúc'' của Tôn Trung Sơn và bước đầu nhận thấy trong đó nhiều tư tưởng tiến bộ , tích cực,phù hợp với xu thế thời đại và có thể vận dụng được vào cách mạng Việt Nam, đó là quan điểm về dân tộc, dân quyền và dân sinh.

\paragraph{Kết luận:}
Cách thức tiếp thu của Hồ Chí Minh: tiếp thu trên tinh thần biện chứng, có chọn lọc, không rập khuôn máy móc, vận dụng sáng tạo và phát triển vào điều kiện cụ thể của Việt Nam. Ưu điểm được Hồ Chí Minh tiếp thu, cương quyết loại bỏ nhược điểm ra khỏi tư tưởng của mình, đồng thời những điểm chưa phù hợp được HCM cải biến cho phù hợp. Có thể nêu ra một số phạm trù của Nho giáo được Hồ Chí Minh sử dụng như Nhân, Nghĩa, Trí, Dũng, Liêm, Trung, Hiếu, v.v... Việc Hồ Chí Minh cải tạo các phạm trù của Nho giáo thể hiện rõ nhất ở hai phạm trù Trung và Hiếu. Hồ Chí Minh viết: ``Đạo đức, ngày trước thì chỉ trung với vua, hiếu với cha mẹ. Ngày nay, thời đại mới, đạo đức cũng phải mới. Phải trung với nước. Phải hiếu với toàn dân, với đồng bào''. Rõ ràng, ở Hồ Chí Minh, nếu chữ Trung mang một nội hàm hoàn toàn mới, từ Trung với vua trở thành Trung với nước, thì chữ Hiếu lại được mở rộng trên cơ sở phổ quát hoá đạo đức cá nhân, trong đó gốc của Hiếu với Dân phải là Hiếu với cha mẹ.

\subsubsection{Giá trị tinh hoa văn hóa nhân loại đối với sinh viên trong giai đoạn hiện nay}

Trong định hướng phát triển đất nước giai đoạn 2021 - 2030, Đảng xác định: ``Phát triển con người Việt Nam tiên tiến, đậm đà bản sắc dân tộc để văn hóa thực sự trở thành sức mạnh nội sinh, động lực phát triển đất nước và bảo vệ Tổ quốc''. Trong bối cảnh toàn cầu hóa và hội nhập quốc tế hiện nay, con người Việt Nam phát triển toàn diện phải là con người có văn hóa, thấm nhuần bản sắc văn hóa dân tộc Việt Nam, có năng lực tiếp thu tinh hoa văn hóa nhân loại, làm cho kho tàng ấy giàu có hơn, phong phú hơn, cũng là một cách để khẳng định mình trong thế giới rộng lớn. Trong bối cảnh toàn cầu hóa và hội nhập hóa quốc tế hiện nay, khi mà nhân loại đang lo âu về sự đánh mất chính mình trong "thế giới phẳng" và nguy cơ đồng phục văn hóa, chúng ta càng thấy rõ hơn về việc ứng xử linh hoạt với các giá trị của tinh hoa văn hóa nhân loại, góp phần bồi đắp những thiếu hụt cho mỗi bên và tạo nên sự đa dạng văn hóa.

Hiện nay nhiều nguồn văn hóa đang du nhập vào nước ta khiến nhiều người dần thay đổi nhận thức và đánh mất đi những bản sắc, giá trị dân tộc. Hơn lúc nào hết, lúc này chúng ta, nhất là thế hệ trẻ "xung kích" để bảo vệ những giá trị truyền thống văn hóa, để đất nước chúng ta vẫn giữ nguyên được bản sắc, hòa nhập chứ không hòa tan. Để làm được điều đó, mỗi sinh viên chúng ta cần phải tự mình phấn đấu, rèn luyện, tự trau dồi cho bản thân những kỹ năng cần thiết, không ngừng nâng cao trình độ chuyên môn, nghiệp vụ, nỗ lực rèn luyện vì lợi ích chung của cộng đồng và vì chính sự phát triển của bản thân. Quan trọng hơn, thế hệ trẻ cần xây dựng bản lĩnh văn hóa, sẵn sàng đấu tranh với những hoạt động, sản phẩm văn hóa không lành mạnh, chủ động, sáng tạo và linh hoạt trong việc tổ chức các hoạt động định hướng cho sinh viên tiếp thu những mặt tích cực, tiên tiến của văn hóa hiện đại.

\cleardoublepage

\subsection{Phân tích tinh hoa văn hóa Phương Tây trong việc hình thành Tư tưởng Hồ Chí Minh, Từ đó, liên hệ với việc tiếp thu giá trị văn hóa phương Tây trong giai đoạn hiện nay.}

\subsubsection{Phân tích tinh hoa văn hóa Phương Tây trong việc hình thành Tư tưởng Hồ Chí Minh}

Ngay từ khi còn học ở Trường tiểu học Pháp-bản xứ ở thành phố Vinh (1905), Hồ Chí Minh đã quan tâm tới khẩu hiệu nổi tiếng của Đại Cách mạng Pháp năm 1789: Tự do - Bình đẳng - Bác ái. Đi sang phương Tây, Người quan tâm tìm hiểu những khẩu hiệu nổi tiếng đó trong các cuộc cách mạng tư sản ở Anh, Pháp, Mỹ. Người đã kế thừa, phát triển những quan điểm nhân quyền, dân quyền trong Bản Tuyên ngôn Độc lập năm 1776 của Mỹ, Bản Tuyên ngôn Nhân quyền và Dân quyền năm 1791 của Pháp và đề xuất quan điểm về quyền mưu cầu độc lập, tự do, hạnh phúc của các dân tộc trong thời đại ngày nay. Trước khi ra đi tìm đường cứu nước, Hồ Chí Minh đã tiếp xúc và chịu ảnh hưởng của văn hóa Pháp, chú ý đến lý tưởng Tự do - Bình đẳng - Bác ái của Cách mạng Pháp (1789), phát triển những quan điểm về dân chủ, quyền độc lập dân tộc trong bản Tuyên ngôn độc lập của nước Mỹ (1776), về nhân quyền và dân quyền trong Tuyên ngôn Nhân quyền và Dân quyền của nước Pháp (1791).

Trong thời gian Hồ Chí Minh sống và hoạt động ở phương Tây, Người trực tiếp nghiên cứu tư tưởng nhân văn, dân chủ và nhà nước pháp quyền của các nhà khai sáng phương Tây như Vônte, Rútxô, Môngtétxkiơ, tìm hiểu chủ nghĩa Tam dân của Tôn Trung Sơn, Trung Quốc; v.v..; thích đọc sách văn học của Shakespeare bằng tiếng Anh, Lỗ Tấn bằng tiếng Trung Hoa, Hugo, Zola bằng tiếng Pháp; hai nhà văn Anatole France và Léon Tolstoi ``có thể nói là những người đỡ đầu văn học'' cho Hồ Chí Minh. Ngoài ra, Hồ Chí Minh còn tham gia các hoạt động chính trị, nghiên cứu lý luận, kinh tế, văn hóa, v.v., đồng thời tiếp thu tư tưởng của Thiên Chúa giáo trong quá trình hình thành tư tưởng của minh, tiêu biểu nhất là tinh thần bác ái, yêu thương con người. Cụ thể:

\paragraph{Tư tưởng tự do, bình đẳng, bác ái:}
Qua các tác phẩm của nhà khai sáng Pháp (Voltaire, Rousso, Montesquieu): tiếp thu tinh thần phê phán chế độ chuyên chế, độc tài và khắc họa được hình ảnh con người thiết tha yêu tự do, khát khao đời sống bình đẳng, bác ái. Đồng thời phê phán sự khác nhau giữa yếu tố tiến bộ, những hình ảnh, giá trị đẹp đẽ của tư tưởng và yếu tố thực tế, mang tính thời sự, tính lịch sử lúc bấy giờ. Những kẻ mệnh danh là người phất cao lá cờ ``Tự do - Bình đẳng - Bác ái'' lại đang làm trái ngược lý tưởng đó ở khắp nơi trên đất Đông Dương. Người tiếp thu tư tưởng, giá trị dân chủ tư sản qua thực tế cuộc sống, điều đó giúp Người có điều kiện hiểu rõ bản chất của bọn thực dân với cái gọi là ``khai hóa thuộc địa''.

\paragraph{Tuyên ngôn nhân quyền và dân quyền của Đại cách mạng tư sản Pháp năm 1791:}
Tiếp thu các quyền như là quyền cá nhân và quyền tập thể của tất cả các giai cấp là bình đẳng, không thể chuyển nhượng và bất khả xâm phạm với mọi mục đích. Người khẳng định: ``Tất cả các dân tộc trên thế giới đều sinh ra bình đẳng; dân tộc nào cũng có quyền sống, quyền sung sướng và quyền tự do''. Việc đề cao tư tưởng nhân quyền và dân quyền như trên chính là để khẳng định: mục tiêu của Cách mạng Việt Nam hoàn toàn phù hợp với khuôn khổ pháp lý quốc tế, với ``lẽ phải'' thông thường và là điều ``không ai có thể chối cãi được''.
Tuyên ngôn độc lập của nước Mỹ (1776): tiếp thu các giá trị về quyền sống, quyền tự do, quyền mưu cầu hạnh phúc;
sự bình đẳng.

\paragraph{Tư tưởng Thiên chúa giáo:}
Tiếp thu lòng nhân ái và đức hy sinh. Hồ Chí Minh đã kế thừa, đề cao những mặt tốt, vận dụng sáng tạo những mặt tốt ấy để làm phong phú thêm tư tưởng của Người về đại đoàn kết, cả đoàn kết toàn dân trong quốc gia dân tộc và đoàn kết quốc tế. Đề cao, động viên đồng bào Thiên chúa giáo tham gia khối để tăng thêm lực lượng cho cách mạng; Vạch mặt bọn thực dân, đế quốc xâm lược lợi dụng tôn giáo, giả danh Chúa vào mục đích, âm mưu ``chia để trị'': gây thù hằn giữa cộng sản với tôn giáo; gây thù hằn giữa đồng bào lương với đồng bào giáo; thù hằn dân tộc này với dân tộc khác... để phục vụ cho mục tiêu xâm lược và chống cộng sản của chúng. Đạo thiên chúa nói riêng và tôn giáo nói chung cùng đồng hành với dân tộc, với cách mạng nước ta.

\subsubsection{Giá trị của việc tiếp thu văn hóa Phương Tây trong giai đoạn hiện nay:}

Quan điểm của Đảng ta: Trong định hướng phát triển đất nước giai đoạn 2021 - 2030, Đảng xác định: "Phát triển con người Việt Nam tiên tiến, đậm đà bản sắc dân tộc để văn hóa thực sự trở thành sức mạnh nội sinh, động lực phát triển đất nước và bảo vệ Tổ quốc". Trong bối cảnh toàn cầu hóa và hội nhập quốc tế hiện nay, con người Việt Nam phát triển toàn diện phải là con người có văn hóa, thấm nhuần bản sắc văn hóa dân tộc Việt Nam, có năng lực tiếp thu tinh hoa văn hóa nhân loại, làm cho kho tàng ấy giàu có hơn, phong phú hơn, cũng là một cách để khẳng định mình trong thế giới rộng lớn. Trong bối cảnh toàn cầu hóa và hội nhập hóa quốc tế hiện nay, khi mà nhân loại đang lo âu về sự đánh mất chính mình trong "thế giới phẳng" và nguy cơ đồng phục văn hóa, chúng ta càng thấy rõ hơn về việc ứng xử linh hoạt với các giá trị của tinh hoa văn hóa nhân loại, góp phần bồi đắp những thiếu hụt cho mỗi bên và tạo nên sự đa dạng văn hóa.

Tiếp thu một cách chọn lọc những giá trị văn hóa phương Tây: Phong cách sống của người phương Tây thiên về sự tự do, tự khám phá giá trị sống; con người ưa sự xê dịch, tìm kiếm các trải nghiệm mới mẻ, họ thích sử dụng các mô hình lập luận, tranh biện, logic, thuật ngữ trong khi trò chuyện. Tuy nhiên, người phương Tây sống theo chủ nghĩa duy vật, đề cao danh tiếng và thành tựu, gắn tiền bạc với sự xa xỉ và giàu có, tôn sùng lợi nhuận. Họ hứng thú và thao tác với công nghệ nhanh hơn là giao tiếp trong đời thực, đồng thời có quan niệm về tình yêu, giá trị hôn nhân khác độc đáo, khác so với chúng ta. Khi đã nhận thức được những giá trị trong văn hóa phương Tây, cần có sự tiếp thu một cách chọn lọc, biết tận dụng những điều tốt, loại bỏ những điều chưa tốt, đồng thời kết hợp hài hòa với truyền thống văn hóa dân tộc ta.

Giá trị của việc tiếp thu văn hóa phương Tây: làm giàu vốn tri thức, kết hợp yếu tố truyền thống và yếu tố hiện đại, phù hợp với các giá trị truyền thống dân tộc, thích nghi với quá trình giao lưu, hội nhập văn hóa. Nhìn chung, trong quá trình toàn cầu hóa, văn hóa, tư duy, lối sống... của phương Tây cũng đem lại những lợi ích nhất định cho Việt Nam, như về các mặt kinh tế, thu nhập quốc dân, mức sống, tiện nghi vật chất ... nhưng mặt khác nó cũng đặt ra vấn đề cần giải quyết. Đó là đạo lý, giá trị truyền thống, bản sắc văn hóa dân tộc đang thật sự bị xâm hại. Trước nguy cơ có thật này, cần có ngay những giải pháp thích hợp để chỉnh đốn lại những nề nếp truyền thống, đồng thời tìm cách phát huy cái hay, cái đẹp, cái độc đáo của bản sắc Việt Nam trên đường hội nhập vào kho tàng văn hóa toàn cầu.
\cleardoublepage