\section{Quan điểm Hồ Chí Minh về độc lập dân tộc.}

\subsection{Phân tích tư tưởng Hồ Chí Minh về độc lập dân tộc và trách nhiệm của sinh viên trong việc bảo vệ độc lập dân tộc.}

\subsubsection{Tư tưởng Hồ Chí Minh về độc lập dân tộc:}

\paragraph{Độc lập, tự do là quyền thiêng liêng, bất khả xâm phạm của tất cả các dân tộc:}
Lịch sử dựng nước và giữ nước của dân tộc Việt Nam từ ngàn xưa đến nay gắn liền với truyền thống yêu nước, đấu tranh chống giặc ngoại xâm. Điều đó nói lên một khát khao to lớn của dân tộc ta là có được một nền độc lập cho dân tộc, tự do cho nhân dân và đó cũng là một giá trị tinh thần thiêng liêng, bất hủ của dân tộc mà Hồ Chí Minh là hiện thân cho tinh thần ấy. Người nói rằng, cái mà tôi cần nhất trên đời là đồng bào tôi được tự do, Tổ quốc tôi được độc lập.

Tại Hội Nghị ở Vécxây (Pháp) năm 1919, Hồ Chí Minh đã gửi tới Hội nghị bản Yêu sách của nhân dân An Nam, với hai nội dung chính là đòi quyền bình đẳng về mặt pháp lý và đòi các quyền tự do, dân chủ. Tuy nhiên bản yêu sách không được Hội nghị chấp nhận nhưng qua đó cho thấy lần đầu tiên, tư tưởng Hồ Chí Minh về quyền của các dân tộc thuộc địa mà trước hết là quyền bình đẳng và tự do đã hình thành.

Trong Chánh cương vắn tắt của Đảng năm 1930, Hồ Chí Minh cũng đã xác định mục tiêu chính trị của Đảng là đánh đổ đế quốc chủ nghĩa Pháp và bọn phong kiến và làm cho nước Nam được hoàn toàn độc lập. Cách mạng Tháng Tám năm 1945 thành công, trong Tuyên ngôn Độc lập, Hồ Chí Minh thay mặt Chính phủ lâm thời trịnh trọng tuyên bố trước quốc dân đồng bào và thế giới rằng: “Nước Việt Nam có quyền hưởng tự do và độc lập, và sự thực đã thành một nước tự do và độc lập. Toàn thể dân Việt Nam quyết đem tất cả tinh thần và lực lượng, tính mạng và của cải để giữ vững quyền tự do và độc lập ấy.”

Ý chí và quyết tâm trên còn được thể hiện trong hai cuộc kháng chiến chống Pháp và chống Mỹ. Khi thực dân Pháp tiến hành xâm lược Việt Nam lần thứ hai, trong Lời kêu gọi toàn quốc kháng chiến ngày 19-12-1946, Người ra lời hiệu triệu, thể hiện quyết tâm sắt đá, bảo vệ cho bằng được nền độc lập dân tộc - giá trị thiêng liêng mà nhân dân Việt Nam mới giành được: “Không! Chúng ta thà hy sinh tất cả, chứ nhất định không chịu mất nước, nhất định không chịu làm nô lệ”.

Năm 1965, đế quốc Mỹ tăng cường mở rộng chiến tranh ở Việt Nam. Trong hoàn cảnh khó khăn, chiến tranh ác liệt đó, Hồ Chí Minh đã nêu lên một chân lý thời đại, một tuyên ngôn bất hủ của các dân tộc khao khát nền độc lập, tự do trên thế giới “Không có gì quý hơn độc lập, tự do”. Với tư tưởng trên của Hồ Chí Minh, nhân dân Việt Nam đã anh dũng chiến đấu, đánh thắng đế quốc Mỹ xâm lược, buộc chúng phải ký kết Hiệp định Paris, cam kết tôn trọng các quyền dân tộc cơ bản của nhân dân ta.

\paragraph{Độc lập dân tộc phải gắn liền tự do, cơm no, áo ấm và hạnh phúc của nhân dân:}
Theo Hồ Chí Minh, độc lập dân tộc phải gắn với tự do của nhân dân. Người đánh giá cao học thuyết “Tam dân” của Tôn Trung Sơn về độc lập và tự do: dân tộc độc lập, dân quyền tự do và dân sinh hạnh phúc. Và bằng lý lẽ đầy thuyết phục, trong khi viện dẫn bản Tuyên ngôn Nhân quyền và Dân quyền của Cách mạng Pháp năm 1791, Hồ Chí Minh khẳng định dân tộc Việt Nam đương nhiên cũng phải được tự do và bình đẳng về quyền lợi: “Đó là lẽ phải không ai chối cãi được”. Người nói: “Nước độc lập mà dân không hưởng hạnh phúc tự do, thì độc lập cũng chẳng có nghĩa lý gì”. Ngoài ra, độc lập cũng phải gắn với cơm no, áo ấm và hạnh phúc của nhân dân. Ngay sau thắng lợi của Cách mạng Tháng Tám năm 1945 trong hoàn cảnh nhân dân đói rét, mù chữ. . . , Hồ Chí Minh yêu cầu phải cố gắng để cho nhân dân ai cũng có cái ăn cái mặc, ai cũng có chỗ ở và được học hành.

Có thể thấy rằng, trong suốt cuộc đời hoạt động cách mạng của Hồ Chí Minh, Người luôn coi độc lập gắn liền với tự do, cơm no, áo ấm cho nhân dân, như Người từng bộc bạch đầy tâm huyết: “Tôi chỉ có một sự ham muốn, ham muốn tột bậc, là làm sao cho nước ta hoàn toàn độc lập, dân ta được hoàn toàn tự do, đồng bào ai cũng có cơm ăn áo mặc, ai cũng được học hành”.

\paragraph{Độc lập dân tộc phải là nền độc lập thật sự, hoàn toàn và triệt để:}
Trong quá trình đi xâm lược các nước, bọn thực dân đế quốc hay dùng chiêu bài mị dân, thành lập các chính phủ bù nhìn bản xứ, tuyên truyền cái gọi là “độc lập tự do” giả hiệu cho nhân dân các nước thuộc địa nhưng thực chất là nhằm che đậy bản chất “ăn cướp” và “giết người” của chúng.

Theo Hồ Chí Minh, độc lập dân tộc phải là độc lập thật sự, hoàn toàn và triệt để trên tất cả các lĩnh vực. Người nhấn mạnh: độc lập mà người dân không có quyền tự quyết về ngoại giao, không có quân đội riêng, không có nền tài chính riêng..., thì độc lập đó chẳng có ý nghĩa gì. Trên tinh thần đó và trong hoàn cảnh đất nước ta sau Cách mạng Tháng Tám còn gặp nhiều khó khăn, Người đã thay mặt Chính phủ ký với đại diện Chính phủ Pháp Hiệp định Sơ bộ ngày 6-3-1946, theo đó: “Chính phủ Pháp công nhận nước Việt Nam Dân chủ Cộng hoà là một quốc gia tự do có Chính phủ của mình, Nghị viện của mình, quân đội của mình, tài chính của mình”.

\paragraph{Độc lập dân tộc gắn liền với thống nhất và toàn vẹn lãnh thổ:}
Trong lịch sử đấu tranh giành độc lập dân tộc, dân tộc ta luôn đứng trước âm mưu chia cắt đất nước của kẻ thù. Thực dân Pháp khi xâm lược nước ta đã chia đất nước ta ra ba kỳ, mỗi kỳ có chế độ cai trị riêng. Trong hoàn cảnh đó, trong bức Thư gửi đồng bào Nam Bộ (1946), Hồ Chí Minh khẳng định: “Đồng bào Nam Bộ là dân nước Việt Nam. Sông có thể cạn, núi có thể mòn, song chân lý đó không bao giờ thay đổi”. Hiệp định Giơnevơ năm 1954 được ký kết, đất nước Việt Nam tạm thời bị chia cắt làm hai miền, Hồ Chí Minh tiếp tục kiên trì đấu tranh để thống nhất Tổ quốc. Trong Di chúc, Người cũng đã thể hiện niềm tin tuyệt đối vào sự thắng lợi của cách mạng, vào sự thống nhất nước nhà: "Đế quốc Mỹ nhất định phải cút khỏi nước ta. Tổ quốc ta nhất định sẽ thống nhất. Đồng bào Nam Bắc nhất định sẽ sum họp một nhà”. Có thể khẳng định rằng tư tưởng độc lập dân tộc gắn liền với thống nhất Tổ quốc, toàn vẹn lãnh thổ là tư tưởng xuyên suốt trong cuộc đời hoạt động cách mạng của Hồ Chí Minh.

\subsubsection{Trách nhiệm của sinh viên trong việc bảo vệ độc lập dân tộc}

\paragraph{Nhận thức về độc lập dân tộc:}
Hiểu rõ hơn về tư tưởng độc lập dân tộc và vai trò của Hồ Chí Minh trong việc đưa đất nước Việt Nam đến với độc lập tự do. Bên cạnh đó, theo tư tưởng của Hồ Chí Minh, độc lập dân tộc là quyền của mỗi quốc gia và mỗi dân tộc, là một giá trị cốt lõi của con người, đóng vai trò quan trọng trong việc đảm bảo sự phát triển và tiến bộ của xã hội. Chính vì thế mà sinh viên cần đưa ra những quan điểm chính xác về độc lập dân tộc, sinh viên cần phải nắm vững các nguyên tắc và giá trị của độc lập dân tộc, bao gồm cả tôn trọng và bảo vệ quyền tự quyết của mỗi quốc gia và mỗi dân tộc sự công bằng và bình đẳng cho tất cả các thành viên trong cộng đồng dân tộc. Sinh viên cần đặt sự tôn trọng và bảo vệ quyền con người lên hàng đầu, đảm bảo sự công bằng và bình đẳng cho tất cả các thành viên trong cộng đồng dân tộc. Đồng thời sinh viên cần phải luôn tìm cách đối thoại và hợp tác với các quốc gia và dân tộc khác trong khu vực và trên thế giới để xây dựng một thế giới hòa bình chính trị ổn định và phát triển bền vững.

\paragraph{Trách nhiệm của sinh viên trong việc bảo vệ độc lập dân tộc thể hiện như thế nào}
Sinh viên cần có tri thức hiểu biết về những vấn đề có liên quan đến sự nghiệp bảo vệ Tổ quốc, bao gồm hiểu biết về đất nước và con người các dân tộc và tôn giáo ở Việt Nam, hiểu biết về lịch sử truyền thống và bản sắc văn hóa dân tộc, hiểu biết về Đảng Cộng sản, về nhà nước xã hội chủ nghĩa của dân do dân và vì dân về chế độ xã hội chủ nghĩa mà chúng ta đang xây dựng.

Sinh viên sinh viên cần phải cảnh giác tích cực trong đấu tranh với những hành động sai trái, không để các thế lực thù địch, các phần tử chống đối lợi dụng mình để thực hiện diễn biến hòa bình, phát hiện những tổ chức người có hành vi tuyên truyền lôi kéo sinh viên tham gia các hoạt động trái quy định của Pháp luật, nhằm chống lại Đảng Nhà nước để bảo báo cho lãnh đạo của trường chính quyền và các cơ quan bảo vệ pháp luật biết.

Sinh viên cần tích cực tự giác tham gia các hoạt động cụ thể để bảo vệ an ninh quốc gia, giữ gìn trật tự và an toàn xã hội, tham gia xây dựng nếp sống văn minh trật tự ngay trong trường học ký túc xá và khu vực dân cư mà mình sinh sống, bảo vệ môi trường, giúp đỡ các cơ quan chuyên trách trong bảo vệ an ninh quốc gia, giữ gìn trật tự an toàn xã hội

Sinh viên cần tăng cường rèn luyện thể lực, học tập tốt môn học giáo dục quốc phòng an ninh, góp phần chuẩn bị cho lực lượng sẵn sàng chiến đấu bảo vệ tổ quốc và giữ gìn trật tự an toàn xã hội

Sinh viên cần tích cực học tập để nâng cao hiểu biết về toàn vẹn lãnh thổ và tuyên truyền những điều đúng đắn cho mọi người xung quanh. Mỗi sinh viên phải nắm chắc đường lối quan điểm của Đảng, chính sách pháp luật của nhà nước về xây dựng nền quốc phòng toàn dân, an ninh nhân dân.