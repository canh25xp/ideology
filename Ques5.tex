\section{Tư tưởng Hồ Chí Minh về vai trò của đại đoàn kết dân tộc}

\subsection{Phân tích tư tưởng Hồ Chí Minh về xây dựng khối đại đoàn kết dân tộc. Trách nhiệm của sinh viên trong việc xây dựng khối đại đoàn kết toàn dân.}

\subsubsection{Quan điểm của Hồ Chí Minh về vai trò của đại đoàn kết dân tộc thể hiện ở các vấn đề chính sau:}

\paragraph{Thứ nhất, đại đoàn kết dân tộc là vấn đề có ý nghĩa chiến lược, quyết định thành công của cách mạng:}
Trong tư tưởng Hồ Chí Minh, đại đoàn kết toàn dân tộc không phải là sách lược hay thủ đoạn chính trị mà là chiến lược lâu dài, nhất quán của cách mạng Việt Nam.

Đây là vấn đề mang tính sống còn của dân tộc Việt Nam nên chiến lược này được duy trì cả trong cách mạng dân tộc dân chủ nhân dân và cách mạng xã hội chủ nghĩa. Trong mỗi giai đoạn cách mạng, trước những yêu cầu và nhiệm vụ khác nhau, chính sách và phương pháp tập hợp đại đoàn kết có thể và cần thiết phải điều chỉnh cho phù hợp với từng đối tượng khác nhau song không bao giờ được thay đổi chủ trương đại đoàn kết toàn dân tộc, vì đó là nhân tố quyết định sự thành bại của cách mạng.

Từ thực tiễn xây dựng khối đại đoàn kết toàn dân tộc, Hồ Chí Minh đã khái quát thành nhiều luận điểm mang tính chân lý về vai trò và sức mạnh của khối đại đoàn kết toàn dân tộc: “Đoàn kết là sức mạnh của chúng ta”, “Đoàn kết là một lực lượng vô địch của chúng ta để khắc phục khó khăn, giành lấy thắng lợi”, “Đoàn kết là sức mạnh, đoàn kết là thắng lợi”, “Đoàn kết là sức mạnh, là then chốt của thành công” “Bây giờ còn một điểm rất quan trọng, cũng là điểm mẹ. Điểm này mà thực hiện tốt thì đẻ ra con cháu đều tốt: Đó là đoàn kết”. Người đã đi đến kết luận:
\begin{center}
    “Đoàn kết, đoàn kết, đại đoàn kết
    
    Thành công, thành công, đại thành công”    
\end{center}

\paragraph{Thứ hai, đại đoàn kết toàn dân tộc là một mục tiêu, nhiệm vụ hàng đầu của cách mạng Việt Nam:}
Đối với Hồ Chí Minh, đại đoàn kết không chỉ là khẩu hiệu chiến lược mà còn là mục tiêu lâu dài của cách mạng. Đảng là lực lượng lãnh đạo cách mạng Việt Nam nên tất yếu đại đoàn kết toàn dân tộc phải được xác định là nhiệm vụ hàng đầu của Đảng và nhiệm vụ này phải được quán triệt trong tất cả mọi lĩnh vực, từ đường lối, chủ trương, chính sách, tới hoạt động thực tiễn của Đảng. Trong lời kết thúc buổi ra mắt Đảng Lao động Việt Nam ngày 3-3-1951, Hồ Chí Minh tuyên bố: “Mục đích của Đảng Lao động Việt Nam có thể gồm trong tám chữ là: ĐOÀN KẾT TOÀN D N, PHỤNG SỰ TỔ QUỐC”.

Cách mạng là sự nghiệp của quần chúng, do quần chúng và vì quần chúng. Đại đoàn kết là yêu cầu khách quan của sự nghiệp cách mạng, là đòi hỏi khách quan của quần chúng nhân dân trong cuộc đấu tranh tự giải phóng bởi nếu không đoàn kết thì chính họ sẽ thất bại trong cuộc đấu tranh vì lợi ích của chính mình. Nhận thức rõ điều đó, Đảng Cộng sản phải có sứ mệnh thức tỉnh, tập hợp, hướng dẫn quần chúng, chuyển những nhu cầu, những đòi hỏi khách quan, tự phát của quần chúng thành những đòi hỏi tự giác, thành hiện thực có tổ chức trong khối đại đoàn kết, tạo thành sức mạnh tổng hợp trong cuộc đấu tranh vì độc lập của dân tộc, tự do cho nhân dân và hạnh phúc cho con người.

Tóm lại, đại đoàn kết dân tộc là mục tiêu, nhiệm vụ của Nhân dân, Đảng Cộng sản Việt Nam và Nhà nước Việt Nam.

\subsubsection{Trách nhiệm của sinh viên trong việc xây dựng khối đại đoàn kết toàn dân.}

Trước hết sinh viên cần nhận thức rõ về vai trò của đại đoàn kết dân tộc, quán triệt tư tưởng đại đoàn kết là vấn đề sống còn của dân tộc, từ đó ý thức được trách nhiệm của bản thân trong việc xây dựng khối đại đoàn kết dân tộc, phát huy tính năng động của bản thân, tinh thần tự lực tự cường, vượt qua mọi thách thức.

Sinh viên cung cần hiểu được về vị trí, vai trò và tầm quan trọng của tư tưởng Hồ Chí Minh về đại đoàn kết toàn dân tộc gắn với thực hiện Chỉ thị số 05- CT/TW của Bộ chính trị về “Tăng cường khối đại đoàn kết toàn dân tộc, xây dựng Đảng và hệ thống chính trị trong sạch, vững mạnh theo tư tưởng, đạo đức, phong cách hồ Chí Minh”. Đại đoàn kết là sự nghiệp của cả dân tộc, của cả hệ thống chính trị mà hạt nhân lãnh đạo là Đảng Cộng sản Việt Nam, được thực hiện bằng nhiều hình thức; trong đó chủ trương, đường lối của Đảng, chính sách, pháp luật của nhà nước có ý nghĩa quan trọng hàng đầu.

Sinh viên cần gương mẫu thực hiện nghĩa vụ công dân: tuyên truyền, vận động gia đình và nhân dân trên địa bàn thực hiện tốt chính sách, chủ trương của Đảng và Nhà nước; bảo vệ quyền lợi, lợi ích hợp pháp của công dân, phát huy dân chủ, giữ gìn kỷ cương, chống quan liêu, tham nhũng, lãng phí.

Trong đời sống, sinh viên cần tích cực tham gia các tổ chức chính trị xã hội nhằm tuyên truyền cho mọi người về tư tưởng đại đoàn kết toàn dân (mặt trận tổ quốc Việt Nam, Hội sinh viên , Đoàn thanh niên...)
